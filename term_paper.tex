\documentclass[a4paper, 12pt]{article}

\usepackage{mathtext}
\usepackage[T2A]{fontenc}
\usepackage[utf8]{inputenc}
\usepackage[russian]{babel}

\usepackage{amsmath}
\usepackage{titlesec}
\usepackage{scrextend}
\usepackage{graphicx}
\usepackage{tikz}
\usetikzlibrary{shapes.misc}
\usepackage{pdflscape}
\usepackage{float}
\usepackage{pgfplots}

\DeclareSymbolFont{T2Aletters}{T2A}{cmr}{m}{it}
\graphicspath{ {./images/} }
\pgfplotsset{width=10cm,compat=1.9}

% Установки для отрисовки решеток кодера
\tikzstyle{lightedge}=[dashed]
\tikzstyle{mainedge}=[solid]
\tikzstyle{activeedge}=[green, very thick]
\tikzstyle{inputBit}=[rectangle,fill=red, text=white]
\tikzstyle{outputBit}=[rectangle,fill=blue, text=white]
\tikzstyle{pointer}=[orange,->,dashed]
\tikzstyle{highlight}=[circle,fill=blue,text=white,scale=0.7]

\newcounter{ctra}
\newcommand{\trellisEdges}[2]{
  \setcounter{ctra}{#2}
  \pgfmathtruncatemacro{\xplusone}{#1 + 1}
  \ifodd\value{ctra}
      \draw[mainedge] (s#1#2) -- (s\xplusone2);
  \else
      \draw[mainedge] (s#1#2) -- (s\xplusone0);
  \fi
  \ifodd\value{ctra}
      \draw[lightedge] (s#1#2) -- (s\xplusone3);
  \else
      \draw[lightedge] (s#1#2) -- (s\xplusone1);
  \fi
}

% #1=x; #2=y; #3=In; #4=Out
\newcommand{\trellisInOut}[4]{
  \node[inputBit] (in#1) at (#1+0.5,4) {#3};
  \node[outputBit] (out#1) at (#1+0.5,5) {#4};
  \draw[pointer] (in#1) -- (#1+0.5,#2);
}

% #1=x; #2=y; #3=In
\newcommand{\trellisIn}[2]{
  \node[outputBit] (in#1) at (#1+0.5,4) {#2};
}


\author{Анатолий Копыл}
\title{Расчёт основных характеристик цифровой системы связи с использованием квадратурной модуляции}

\begin{document}

% НАЧАЛО ТИТУЛЬНОГО ЛИСТА
\makeatletter
\begin{titlepage}
  \begin{center}
    \hfill \break
    \footnotesize{ФЕДЕРАЛЬНОЕ ГОСУДАРСТВЕННОЕ БЮДЖЕТНОЕ ОБРАЗОВАТЕЛЬНОЕ УЧРЕЖДЕНИЕ}\\ 
    \footnotesize{ВЫСШЕГО ПРОФЕССИОНАЛЬНОГО ОБРАЗОВАНИЯ}\\
    \small{\textbf{«Санкт-Петербургский государственный университет телекоммуникаций им. проф. М. А. Бонч-Бруевича»}}\\
    \hfill \break
    \normalsize{Факультет инфокоммуникационных сетей и систем}\\
    \hfill \break
    \normalsize{Кафедра теоретических основ связи и радиотехники}\\
    \hfill\break
    \hfill \break
    \hfill \break
    \hfill \break
    \large{ \@title }\\
    \hfill \break
    \hfill \break
    \normalsize{Учебная дисциплина <<Теория электрической связи>>}\\
    \hfill \break
    \hfill \break
    \hfill \break
    \normalsize{Курсовая работа}\\
    \hfill \break
    \hfill \break
  \end{center}
    
  \hfill \break
  \hfill \break
  
  \normalsize{
    \hfill\begin{minipage}{\dimexpr\textwidth-6cm}
      Студент группы ИКТО-91 Копыл А. В.\\
      зачетная книжка № 1905141\\\\
      Руководитель \underline{\hspace{4cm}}
    \end{minipage}
  }\\
  \vfill
  \begin{center} Санкт-Петербург 2021 \end{center}
  \thispagestyle{empty} % выключаем отображение номера для этой страницы
\end{titlepage}
\makeatother
% КОНЕЦ ТИТУЛЬНОГО ЛИСТА
  
\newpage

Цель курсовой работы -- изучить и разработать систему цифровой связи, 
оптимальную в отношении флуктуационной помехи и исключающую появления 
межсимвольной помехи.

\section{Структурная схема системы\\цифровой связи}

Система связи предназначена для передачи аналоговых сообщений
по цифровому каналу связи.
\begin{figure}[H]
  \includegraphics[scale=0.5]{struct_scheme}
  \caption{Структурная схема цифровой системы связи}
  \label{fig:struct_scheme}
\end{figure}

В систему входят следующие функциональные узлы с последующими назначениями:
\begin{enumerate}
  \item Источник сообщений -- создает реализации $a(t)$ случайного
  процесса $A(t)$.
  \item Аналого-цифровой преобразователь -- преобразует аналоговый
  сигнал от источника сообщения в последовательность 
  двоичных отсчетов $b(t)$.
  \item Кодер -- включает в цифровой поток от АЦП дополнительные
  символы, предназначенные для повышения помехоустойчивости системы
  связи;
  \item Формирователь модулирующих символов -- служит для получения
  модулирующих сигналов $I(t)$ и $Q(t)$, соответствующих заданному
  виду модуляции;
  \item Сглаживающие формирующие фильтры (СФФ1, СФФ2);
  \item Перемножители -- для получения БМ сигналов: синфазного 
  $I(t)\cos{\omega_Ct}$ и квадратурного $Q(t)\sin{\omega_Ct}$.
  \item Фазовращатель -- для получения второго несущего колебания, 
  ортогонального по отношению к первому;
  \item Генератор гармонических колебаний -- для получения несущего  
  колебания;
  \item Инвертор;
  \item Сумматор -- для объединения синфазного и квадратурного 
  сигналов в единый сигнал с квадратурной модуляцией 
  $S_{КАМ}(t) = I(t)\cos{\omega_Ct} + Q(t)\sin{\omega_Ct}$;
  \item Непрерывный канал -- среда распространения сигнала 
  $S_{КАМ}(t)$;
  \item Демодулятор -- для анализа приходящего сигнала, 
  искаженного помехами, и принятии решения о переданном сообщении; 
  \item Преобразователь параллельного кода в последовательный код --
  для преобразования сигнала с выхода демодулятора в 
  последовательный формат кодовых комбинаций;
  \item Декодер -- для исправления части ошибок, возникших при приёме 
  сообщения $\hat{b}(t)$ вследствие влияния помех; 
  \item Цифро-аналоговый преобразователь -- для восстановления  
  аналоговой формы сигнала $\hat{a}(t)$ из его цифрового представления;
  \item Получатель сообщений.
\end{enumerate}

\section{Исходные данные}
$m=41$
\begin{center}
  \begin{tabular}{ | p{5cm} | p{5cm} | p{5cm} | } 
    \hline
    Предельные уровни аналогового сигнала \(a_{мин}\), \(a_{макс}\) (В) & \(a_{макс}=25,6\) В;\newline\(a_{мин}=-25,6\) В & Внести свои данные \\
    \hline
    Верхняя частота спектра аналогового сигнала \(f_В\) & \(f_В =(1+m\cdot 10^{-2})\cdot 10^4\) & \(f_В =14100\) \\ 
    \hline
    Заданный уровень квантования & \(j=500-3\cdot m\) & 377 \\
    \hline
    Спектральная плотность мощности флуктуационной помехи & 41 & \(N_0=2,3\cdot 10^{-7}\, В^2/Гц\)\\
    \hline
    q -- номер тактового интервала ошибки & \(q=m\mod{3}+1\) & \(q=3\)\\
    \hline
    Вид модуляции & КАМ-16 & \\
    \hline
  \end{tabular}
\end{center}

\section{Расчет составляющих системы цифровой связи}

\subsection{Источник сообщений}
Источник сообщения (ИС) вырабатывает реализации $a(t)$ стационарного
случайного процесса $A(t)$, типа квазибелого шума с параметрами 
$a_{мин}$, $a_{макс}$ и $f_В$. Мгновенные значения сообщения
равновероятны в интервале от значения $a_{мин}$ и до значения 
$a_{макс}$.

Требуется:
\begin{enumerate}
  \item Написать аналитические выражения для плотности вероятности 
  $w(а)$ мгновенных значений сообщения, функции распределения $F(a)$ и
  построить их графики (рис. \ref{fig:prob_plots}).

  \[ w(a)=\frac{1}{a_{макс}-a_{мин}}=\frac1\Delta=\frac{1}{25,6+25,6}=0,02 \]
  \[ F(a)=\int^a_{-\infty}w(a)da=
  \int^a_{a_{мин}}\frac{1}{\Delta}da=
  \begin{cases}
    1, & a > a_{макс}\\
    \frac{a-a_{мин}}{\Delta}, & a_{мин} \leq a \leq a_{макс}\\
    0, & a < a_{мин}
  \end{cases}\]
  где $\Delta = a_{макс}-a_{мин}=51,2\, В$.

  % Графики
  \begin{figure}[H]
    \centering
    \begin{tikzpicture}
      \pgfmathsetmacro{\amin}{-25.6}
      \pgfmathsetmacro{\amax}{25.6}
      \begin{axis}[
        width=6cm,height=4cm,
        axis lines = left,
        xlabel = $a$,
        ylabel = {$F(a)$},
        xmin=-40, xmax=40,
        ymin=0, ymax=1.25,
      ]
        \addplot [
          domain=-40:\amin, 
          color=red,
        ]
        {0};
        \addplot [
          domain=\amin:\amax,
          samples=2,
          color=red,
        ]
        {(x-\amin) / 51.2};
        \addplot [
          domain=\amax:40, 
          color=red,
        ]
        {1};
      \end{axis}
    \end{tikzpicture}%
    \begin{tikzpicture}
      \pgfmathsetmacro{\amin}{-25.6}
      \pgfmathsetmacro{\amax}{25.6}
      \begin{axis}[
        width=6cm,height=4cm,
        axis lines = left,
        xlabel = $a$,
        ylabel = {$w(a)$},
        xmin=-40, xmax=40,
        ymin=0, ymax=0.03,
      ]
        \addplot [
          domain=-40:\amin, 
          color=blue,
        ]
        {0};
        \addplot [
          domain=\amin:\amax,
          samples=2,
          color=blue,
        ]
        {0.02};
        \addplot [
          domain=\amax:40, 
          color=blue,
        ]
        {0};
        \draw [dashed] (axis cs:\amin,0) -- (axis cs:\amin,0.02);
        \draw [dashed] (axis cs:\amax,0) -- (axis cs:\amax,0.02);
      \end{axis}
    \end{tikzpicture}
    \caption{Графики функции распределения и плотности вероятности}
    \label{fig:prob_plots}
  \end{figure}
  \item Рассчитать математическое ожидание $\overline{A(t)}$ и 
  дисперсию $D\{A(t)\}$ сообщения $A(t)$.
  \[ \overline{A(t)}=\int^\infty_{-\infty}a\cdot w(a)da=
  \int^{a_{макс}}_{a_{мин}}a \frac{1}{a_{макс}-a_{мин}} da=
  \frac{a^2}{2\Delta} \Biggr|^{a_{макс}}_{a_{мин}}\! =
  \frac{a_{макс}^2-a_{мин}^2}{2\Delta}=0 \]

  \begin{align*}\begin{split}
    D\{A(t)\}&=\int^\infty_{-\infty}(a-\overline{A(t)})^2 w(a)da=
    \int^{a_{макс}}_{a_{мин}}a^2w(a)da\\
    &=\frac{a^3}{3\Delta}\Biggr|^{a_{макс}}_{a_{мин}}\!
    =\frac{a_\text{min}^2+a_\text{max}a_\text{min}+a_\text{max}^2}{3}
    =218,5
  \end{split}\end{align*}
  \item Написать аналитическое выражение для спектральной плотности
  мощности $G_A(f)$ сообщения $A(t)$ и построить график 
  (рис. \ref{fig:spectr_plot}).
  \[ G_A(f)=\frac{D\{A(t)\}}{2f_В}=\frac{218,5}{2\cdot1,41\cdot 10^4}
  =7,7 \,мВ^2/Гц \]
  \[ G_A(f)=\begin{cases}
    7,7 \,мВ^2/Гц, & |f| \leq f_B\\
    0, & |f| > f_B
  \end{cases} \]
  \begin{figure}[H]
    \centering
    \begin{tikzpicture}
      \pgfmathsetmacro{\fv}{14100}
      \pgfmathsetmacro{\Gaf}{0.0077}
      \begin{axis}[
        width=6cm,height=4cm,
        axis lines = left,
        ylabel = {$G_A(f)$},
        xmin=-\fv*1.5, xmax=\fv*1.5,
        ymin=0, ymax=\Gaf*1.5,
      ]
        \addplot [
          domain=-\fv*1.5:-\fv, 
          color=blue,
        ]
        {0};
        \addplot [
          domain=-\fv:\fv,
          samples=2,
          color=blue,
        ]
        {\Gaf};
        \addplot [
          domain=\fv:\fv*1.5, 
          color=blue,
        ]
        {0};
        \draw [dashed] (axis cs:-\fv,0) -- (axis cs:-\fv,\Gaf);
        \draw [dashed] (axis cs:\fv,0) -- (axis cs:\fv,\Gaf);
      \end{axis}
    \end{tikzpicture}
    \caption{График спектральной плотности мощности.}
    \label{fig:spectr_plot}
  \end{figure}
  \item Найти аналитическое выражение для корреляционной функции
  $B_A(\tau)$ сообщения $A(t)$ и построить график 
  (рис. \ref{fig:coorel_plot}). 
  По форме графика $B_A(\tau)$ определить, 
  является ли сообщение $A(t)$ эргодическим случайным процессом 
  или не является таковым.

  \begin{align*}\begin{split}
    B_A(\tau)&=\int^\infty_{-\infty}\frac{G_A(f)}{2}e^{j2\pi f\tau}df
    =\int^{f_B}_{-f_B}\frac{G_A}{2}\cos{2\pi f\tau}df\\
    &=\frac{G_A}2 \frac{\sin{2\pi f \tau}}{2\pi \tau}\Biggr|^{f_B}_{-f_B} 
    =G_A\frac{\sin{2\pi f_B \tau}}{2\pi\tau}
  \end{split}\end{align*}
  \begin{figure}[H]
    \centering
    \begin{tikzpicture}
      \pgfmathsetmacro{\PI}{3.14159}
      \pgfmathsetmacro{\fv}{14100}
      \pgfmathsetmacro{\Ga}{0.0077}
      \begin{axis}[
        width=10cm,height=6cm,
        axis lines = left,
        ylabel = {$B_A(\tau)$},
        xlabel = {$\tau$},
      ]
        \addplot [
          color=blue,
          samples=100,
          domain=-0.01:0.01,
        ]
        {\Ga*(sin(2*\PI*\fv*x))/(2*\PI*x)};
      \end{axis}
    \end{tikzpicture}
    \caption{График корреляционной функции $B_A(\tau)$.}
    \label{fig:coorel_plot}
  \end{figure}
\end{enumerate}

\subsection{Аналого-цифровой преобразователь}
\[ \Delta t \leq \frac{1}{2f_B}=\frac1 {2\cdot 14100} = 3,546\cdot 10^{-5}\, с \]
\[ f_d=\frac{1}{\Delta t}\geq 2f_B=\frac{1}{3,546\cdot 10^{-5}}=28200 \]
\[ 377_{10}=101111001_2 \]
\[ k=9;\, L=2^9 = 512 \]

\subsection{Кодер}
\begin{center}
  \includegraphics[scale=0.8]{coder}

  \begin{tabular}{ | c | c | c | c | c | c | c | c | c | c | }
    \hline
    Входной сигнал &1&0&1&1&1&1&0&0&1\\
    \hline
    Выходной сигнал &11&10&00&01&10&10&01&11&11\\
    \hline
  \end{tabular}
\end{center}

\subsubsection{Решетка кодера}

%\begin{landscape}
  \begin{figure}[H]
    \begin{tikzpicture}[x=1cm, y=-1cm]
      \node at (-0.5,0) [left] {$s_1=00$};
      \node at (-0.5,1) [left] {$s_2=10$};
      \node at (-0.5,2) [left] {$s_3=01$};
      \node at (-0.5,3) [left] {$s_4=11$};

      % Nodes
      \foreach \x in {0,...,12} {
        \node at (\x,-.7) {$\x$};
        \foreach \y in {0,...,3} {
          \node (s\x\y) at (\x,\y) [circle,fill=black,scale=0.7] {};
        }
      }

      % Edges
      \trellisEdges{0}{0}
      \trellisEdges{1}{0}
      \trellisEdges{1}{1}
      \foreach \x in {2,...,11} {
        \foreach \y in {0,...,3} {
          \trellisEdges{\x}{\y}
        }
      }

      % Inputs and Outputs
      \node at (-0.5,4) [left] {ИС};
      \node at (-0.5,5) [left] {КС};

      \trellisInOut{0}{0.5}{1}{11}
      \trellisInOut{1}{1.5}{0}{10}
      \trellisInOut{2}{1.5}{1}{00}
      \trellisInOut{3}{2}{1}{01}
      \trellisInOut{4}{3}{1}{10}
      \trellisInOut{5}{3}{1}{10}
      \trellisInOut{6}{2.5}{0}{01}
      \trellisInOut{7}{1}{0}{11}
      \trellisInOut{8}{0.5}{1}{11}
      \trellisInOut{9}{1.5}{0}{01}
      \trellisInOut{10}{1}{0}{11}
      \trellisInOut{11}{0}{0}{00}
    \end{tikzpicture}

    \caption{Путь на решетке кодера} \label{fig:coder}
  \end{figure}
%\end{landscape}

Длительность двоичного символа \(T_В\) на выходе кодера:
\[T_В=\frac{\Delta t}{2k}=\frac{3,546\cdot 10^{-5}}{2\cdot 9}=
1,97\cdot 10^{-6}\,с\]

\subsection{Декодер}
По каналу передавался код \(\overline{u}=11 10 00 01 10 10 01 11 11\).
Ошибка произошла на тактовом интервале \(q=3\).
Таким образом, на вход декодера поступает последовательность 
\(\overline{Z}=11 \overset{\times}{0} 0 00 01 10 10 01 11 11\). Крестиком обозначен ошибочно принятый символ.

\subsubsection{Диаграмма декодера}
\begin{tikzpicture}[x=1.2cm, y=-1cm]

  \node at (-0.5,0) [left] {$s_1=00$};
  \node at (-0.5,1) [left] {$s_2=10$};
  \node at (-0.5,2) [left] {$s_3=01$};
  \node at (-0.5,3) [left] {$s_4=11$};

  % Nodes
  \foreach \x in {0,...,9} {
    \node at (\x,-.7) {$\x$};
    \foreach \y in {0,...,3} {
      \node (s\x\y) at (\x,\y) [circle,fill=black,scale=0.7] {};
    }
  }

  \node at (0,0) [highlight] {};
  \node at (1,0) [highlight,label=left:{$2$}] {};
  \node at (1,1) [highlight,label=left:{$0$}] {};

  \node at (2,0) [highlight,label=left:{$0$}] {};
  \node at (2,1) [highlight,label=left:{$2$}] {};
  \node at (2,2) [highlight,label=left:{$1$}] {};
  \node at (2,3) [highlight,label=left:{$1$}] {};

  \node at (3,0) [highlight,label=left:{$\frac{0}{2}$}] {};
  \node at (3,1) [highlight,label=left:{$\frac{2}{0}$}] {};
  \node at (3,2) [highlight,label=left:{$\frac{1}{1}$}] {};
  \node at (3,3) [highlight,label=left:{$\frac{1}{1}$}] {};

  \node at (4,0) [highlight,label=left:{$\frac{1}{1}$}] {};
  \node at (4,1) [highlight,label=left:{$\frac{1}{1}$}] {};
  \node at (4,2) [highlight,label=left:{$\frac{2}{0}$}] {};
  \node at (4,3) [highlight,label=left:{$\frac{0}{2}$}] {};

  \node at (5,0) [highlight,label=left:{$\frac{1}{1}$}] {};
  \node at (5,1) [highlight,label=left:{$\frac{1}{1}$}] {};
  \node at (5,2) [highlight,label=left:{$\frac{0}{2}$}] {};
  \node at (5,3) [highlight,label=left:{$\frac{2}{0}$}] {};

  \node at (6,0) [highlight,label=left:{$\frac{1}{1}$}] {};
  \node at (6,1) [highlight,label=left:{$\frac{1}{1}$}] {};
  \node at (6,2) [highlight,label=left:{$\frac{0}{2}$}] {};
  \node at (6,3) [highlight,label=left:{$\frac{2}{0}$}] {};

  \node at (7,0) [highlight,label=left:{$\frac{1}{1}$}] {};
  \node at (7,1) [highlight,label=left:{$\frac{1}{1}$}] {};
  \node at (7,2) [highlight,label=left:{$\frac{2}{0}$}] {};
  \node at (7,3) [highlight,label=left:{$\frac{0}{2}$}] {};

  \node at (8,0) [highlight,label=left:{$\frac{2}{0}$}] {};
  \node at (8,1) [highlight,label=left:{$\frac{0}{2}$}] {};
  \node at (8,2) [highlight,label=left:{$\frac{1}{1}$}] {};
  \node at (8,3) [highlight,label=left:{$\frac{1}{1}$}] {};

  \node at (9,0) [highlight,label=left:{$\frac{2}{0}$}] {};
  \node at (9,1) [highlight,label=left:{$\frac{0}{2}$}] {};
  \node at (9,2) [highlight,label=left:{$\frac{1}{1}$}] {};
  \node at (9,3) [highlight,label=left:{$\frac{1}{1}$}] {};

  % Edges
  \trellisEdges{0}{0}
  \trellisEdges{1}{0}
  \trellisEdges{1}{1}
  \foreach \x in {2,...,8} {
    \foreach \y in {0,...,3} {
      \trellisEdges{\x}{\y}
    }
  }

  % Inputs
  \node at (-0.5,4) [left, align=right] {Входная\\пара};

  \trellisIn{0}{11}
  \trellisIn{1}{00}
  \trellisIn{2}{00}
  \trellisIn{3}{01}
  \trellisIn{4}{10}
  \trellisIn{5}{10}
  \trellisIn{6}{01}
  \trellisIn{7}{11}
  \trellisIn{8}{11}
\end{tikzpicture}

Сегмент решетки декодера от $t=0$, до $t=3$.

\begin{tikzpicture}[x=2cm, y=-1cm]

  \node at (-0.5,0) [left] {$s_1=00$};
  \node at (-0.5,1) [left] {$s_2=10$};
  \node at (-0.5,2) [left] {$s_3=01$};
  \node at (-0.5,3) [left] {$s_4=11$};

  % Nodes
  \foreach \x in {0,...,3} {
    \node at (\x,-.7) {$\x$};
    \foreach \y in {0,...,3} {
      \node (s\x\y) at (\x,\y) [circle,fill=black,scale=0.7] {};
    }
  }

  % Edges
  \trellisEdges{0}{0}
  \trellisEdges{1}{0}
  \trellisEdges{1}{1}
  \foreach \x in {2,...,2} {
    \foreach \y in {0,...,3} {
      \trellisEdges{\x}{\y}
    }
  }

  \draw[activeedge] (s00) -- (s10);
  \draw[activeedge] (s00) -- (s11);
  \draw[activeedge] (s10) -- (s20);
  \draw[activeedge] (s11) -- (s22);
  \draw[activeedge] (s20) -- (s30);
  \draw[activeedge] (s22) -- (s31);
  \draw[activeedge] (s11) -- (s23);
  \draw[activeedge] (s23) -- (s32);
  \draw[activeedge] (s23) -- (s33);

  \node at (0,0) [highlight] {};
  \node at (1,0) [highlight,label=left:{$2$}] {};
  \node at (1,1) [highlight,label=left:{$0$}] {};

  \node at (2,0) [highlight,label=left:{$0$}] {};
  \node at (2,1) [highlight,label=left:{$2$}] {};
  \node at (2,2) [highlight,label=left:{$1$}] {};
  \node at (2,3) [highlight,label=left:{$1$}] {};

  \node at (3,0) [highlight,label=left:{$\frac{0}{2}$}] {$\frac23$};
  \node at (3,1) [highlight,label=left:{$\frac{2}{0}$}] {$\frac41$};
  \node at (3,2) [highlight,label=left:{$\frac{1}{1}$}] {$\frac52$};
  \node at (3,3) [highlight,label=left:{$\frac{1}{1}$}] {$\frac52$};

  \node at (2.5,1) [text=red] {$\times$};
  \node at (2.5,0.5) [text=red] {$\times$};
  \node at (2.5,1.5) [text=red] {$\times$};
  \node at (2.5,2) [text=red] {$\times$};
  \node at (1.5,0.5) [text=red] {$\times$};

  % Inputs and Outputs
  \node at (-0.5,4) [left, align=right] {Входная\\пара};

  \trellisIn{0}{11}
  \trellisIn{1}{00}
  \trellisIn{2}{00}
\end{tikzpicture}

Сегмент решетки декодера от $t=0$, до $t=4$.

\begin{tikzpicture}[x=2cm, y=-1cm]

  \node at (-0.5,0) [left] {$s_1=00$};
  \node at (-0.5,1) [left] {$s_2=10$};
  \node at (-0.5,2) [left] {$s_3=01$};
  \node at (-0.5,3) [left] {$s_4=11$};

  % Nodes
  \foreach \x in {0,...,4} {
    \node at (\x,-.7) {$\x$};
    \foreach \y in {0,...,3} {
      \node (s\x\y) at (\x,\y) [circle,fill=black,scale=0.7] {};
    }
  }

  % Edges
  \trellisEdges{0}{0}
  \trellisEdges{1}{0}
  \trellisEdges{1}{1}
  \foreach \x in {2,...,3} {
    \foreach \y in {0,...,3} {
      \trellisEdges{\x}{\y}
    }
  }


  \draw[activeedge] (s00) -- (s11);
  \draw[activeedge] (s11) -- (s22);
  \draw[activeedge] (s22) -- (s31);
  \draw[activeedge] (s11) -- (s23);
  \draw[activeedge] (s23) -- (s32);
  \draw[activeedge] (s23) -- (s33);

  \draw[activeedge] (s31) -- (s43);
  \draw[activeedge] (s32) -- (s41);
  \draw[activeedge] (s32) -- (s40);
  \draw[activeedge] (s33) -- (s42);

  \node at (0,0) [highlight] {};
  \node at (1,0) [highlight,label=left:{$2$}] {};
  \node at (1,1) [highlight,label=left:{$0$}] {};

  \node at (2,0) [highlight,label=left:{$0$}] {};
  \node at (2,1) [highlight,label=left:{$2$}] {};
  \node at (2,2) [highlight,label=left:{$1$}] {};
  \node at (2,3) [highlight,label=left:{$1$}] {};

  \node at (3,0) [highlight,label=left:{$\frac{0}{2}$}] {};
  \node at (3,1) [highlight,label=left:{$\frac{2}{0}$}] {};
  \node at (3,2) [highlight,label=left:{$\frac{1}{1}$}] {};
  \node at (3,3) [highlight,label=left:{$\frac{1}{1}$}] {};

  \node at (2.5,1) [text=red] {$\times$};
  \node at (2.5,0.5) [text=red] {$\times$};
  \node at (2.5,1.5) [text=red] {$\times$};
  \node at (2.5,2) [text=red] {$\times$};
  \node at (1.5,0.5) [text=red] {$\times$};
  \node at (1.5,0) [text=red] {$\times$};
  \node at (0.5,0) [text=red] {$\times$};

  \node at (4,0) [highlight,label=left:{$\frac{1}{1}$}] {$\frac33$};
  \node at (4,1) [highlight,label=left:{$\frac{1}{1}$}] {$\frac33$};
  \node at (4,2) [highlight,label=left:{$\frac{2}{0}$}] {$\frac32$};
  \node at (4,3) [highlight,label=left:{$\frac{0}{2}$}] {$\frac14$};

  \node at (3.5,0) [text=red] {$\times$};
  \node at (3.5,0.5) [text=red] {$\times$};
  \node at (3.5,1.5) [text=red] {$\times$};
  \node at (3.5,3) [text=red] {$\times$};
  \node at (2.5,0) [text=red] {$\times$};

  % Inputs and Outputs
  \node at (-0.5,4) [left, align=right] {Входная\\пара};

  \trellisIn{0}{11}
  \trellisIn{1}{00}
  \trellisIn{2}{00}
  \trellisIn{3}{01}
\end{tikzpicture}

Сегмент решетки декодера от $t=0$, до $t=5$.

\begin{tikzpicture}[x=2cm, y=-1cm]

  \node at (-0.5,0) [left] {$s_1=00$};
  \node at (-0.5,1) [left] {$s_2=10$};
  \node at (-0.5,2) [left] {$s_3=01$};
  \node at (-0.5,3) [left] {$s_4=11$};

  % Nodes
  \foreach \x in {0,...,5} {
    \node at (\x,-.7) {$\x$};
    \foreach \y in {0,...,3} {
      \node (s\x\y) at (\x,\y) [circle,fill=black,scale=0.7] {};
    }
  }

  % Edges
  \trellisEdges{0}{0}
  \trellisEdges{1}{0}
  \trellisEdges{1}{1}
  \foreach \x in {2,...,4} {
    \foreach \y in {0,...,3} {
      \trellisEdges{\x}{\y}
    }
  }


  \draw[activeedge] (s00) -- (s11);
  \draw[activeedge] (s11) -- (s22);
  \draw[activeedge] (s22) -- (s31);
  \draw[activeedge] (s11) -- (s23);
  \draw[activeedge] (s23) -- (s32);
  \draw[activeedge] (s23) -- (s33);

  \draw[activeedge] (s31) -- (s43);
  \draw[activeedge] (s32) -- (s41);
  \draw[activeedge] (s33) -- (s42);

  \draw[activeedge] (s41) -- (s52);
  \draw[activeedge] (s42) -- (s50);
  \draw[activeedge] (s42) -- (s51);
  \draw[activeedge] (s43) -- (s53);

  \node at (0,0) [highlight] {};
  \node at (1,0) [highlight,label=left:{$2$}] {};
  \node at (1,1) [highlight,label=left:{$0$}] {};

  \node at (2,0) [highlight,label=left:{$0$}] {};
  \node at (2,1) [highlight,label=left:{$2$}] {};
  \node at (2,2) [highlight,label=left:{$1$}] {};
  \node at (2,3) [highlight,label=left:{$1$}] {};

  \node at (3,0) [highlight,label=left:{$\frac{0}{2}$}] {};
  \node at (3,1) [highlight,label=left:{$\frac{2}{0}$}] {};
  \node at (3,2) [highlight,label=left:{$\frac{1}{1}$}] {};
  \node at (3,3) [highlight,label=left:{$\frac{1}{1}$}] {};

  \node at (2.5,1) [text=red] {$\times$};
  \node at (2.5,0.5) [text=red] {$\times$};
  \node at (2.5,1.5) [text=red] {$\times$};
  \node at (2.5,2) [text=red] {$\times$};
  \node at (1.5,0.5) [text=red] {$\times$};
  \node at (1.5,0) [text=red] {$\times$};
  \node at (0.5,0) [text=red] {$\times$};

  \node at (4,0) [highlight,label=left:{$\frac{1}{1}$}] {};
  \node at (4,1) [highlight,label=left:{$\frac{1}{1}$}] {};
  \node at (4,2) [highlight,label=left:{$\frac{2}{0}$}] {};
  \node at (4,3) [highlight,label=left:{$\frac{0}{2}$}] {};

  \node at (3.5,0) [text=red] {$\times$};
  \node at (3.5,0.5) [text=red] {$\times$};
  \node at (3.5,1.5) [text=red] {$\times$};
  \node at (3.5,3) [text=red] {$\times$};
  \node at (2.5,0) [text=red] {$\times$};

  \node at (5,0) [highlight,label=left:{$\frac{1}{1}$}] {$\frac43$};
  \node at (5,1) [highlight,label=left:{$\frac{1}{1}$}] {$\frac43$};
  \node at (5,2) [highlight,label=left:{$\frac{0}{2}$}] {$\frac33$};
  \node at (5,3) [highlight,label=left:{$\frac{2}{0}$}] {$\frac51$};

  \node at (4.5,0) [text=red] {$\times$};
  \node at (4.5,0.5) [text=red] {$\times$};
  \node at (4.5,2.5) [text=red] {$\times$};
  \node at (4.5,2) [text=red] {$\times$};
  \node at (3.5,1) [text=red] {$\times$};

  % Inputs and Outputs
  \node at (-0.5,4) [left, align=right] {Входная\\пара};

  \trellisIn{0}{11}
  \trellisIn{1}{00}
  \trellisIn{2}{00}
  \trellisIn{3}{01}
  \trellisIn{4}{10}
\end{tikzpicture}

Сегмент решетки декодера от $t=0$, до $t=6$.

\begin{tikzpicture}[x=1.8cm, y=-1cm]

  \node at (-0.5,0) [left] {$s_1=00$};
  \node at (-0.5,1) [left] {$s_2=10$};
  \node at (-0.5,2) [left] {$s_3=01$};
  \node at (-0.5,3) [left] {$s_4=11$};

  % Nodes
  \foreach \x in {0,...,6} {
    \node at (\x,-.7) {$\x$};
    \foreach \y in {0,...,3} {
      \node (s\x\y) at (\x,\y) [circle,fill=black,scale=0.7] {};
    }
  }

  % Edges
  \trellisEdges{0}{0}
  \trellisEdges{1}{0}
  \trellisEdges{1}{1}
  \foreach \x in {2,...,5} {
    \foreach \y in {0,...,3} {
      \trellisEdges{\x}{\y}
    }
  }


  \draw[activeedge] (s00) -- (s11);
  \draw[activeedge] (s11) -- (s22);
  \draw[activeedge] (s22) -- (s31);
  \draw[activeedge] (s11) -- (s23);
  \draw[activeedge] (s23) -- (s32);
  \draw[activeedge] (s23) -- (s33);

  \draw[activeedge] (s31) -- (s43);
  \draw[activeedge] (s32) -- (s41);
  \draw[activeedge] (s33) -- (s42);

  \draw[activeedge] (s41) -- (s52);
  \draw[activeedge] (s42) -- (s51);
  \draw[activeedge] (s43) -- (s53);

  \draw[activeedge] (s51) -- (s62);
  \draw[activeedge] (s52) -- (s61);
  \draw[activeedge] (s52) -- (s60);
  \draw[activeedge] (s53) -- (s63);

  \node at (0,0) [highlight] {};
  \node at (1,0) [highlight,label=left:{$2$}] {};
  \node at (1,1) [highlight,label=left:{$0$}] {};

  \node at (2,0) [highlight,label=left:{$0$}] {};
  \node at (2,1) [highlight,label=left:{$2$}] {};
  \node at (2,2) [highlight,label=left:{$1$}] {};
  \node at (2,3) [highlight,label=left:{$1$}] {};

  \node at (3,0) [highlight,label=left:{$\frac{0}{2}$}] {};
  \node at (3,1) [highlight,label=left:{$\frac{2}{0}$}] {};
  \node at (3,2) [highlight,label=left:{$\frac{1}{1}$}] {};
  \node at (3,3) [highlight,label=left:{$\frac{1}{1}$}] {};

  \node at (2.5,1) [text=red] {$\times$};
  \node at (2.5,0.5) [text=red] {$\times$};
  \node at (2.5,1.5) [text=red] {$\times$};
  \node at (2.5,2) [text=red] {$\times$};
  \node at (1.5,0.5) [text=red] {$\times$};
  \node at (1.5,0) [text=red] {$\times$};
  \node at (0.5,0) [text=red] {$\times$};

  \node at (4,0) [highlight,label=left:{$\frac{1}{1}$}] {};
  \node at (4,1) [highlight,label=left:{$\frac{1}{1}$}] {};
  \node at (4,2) [highlight,label=left:{$\frac{2}{0}$}] {};
  \node at (4,3) [highlight,label=left:{$\frac{0}{2}$}] {};

  \node at (3.5,0) [text=red] {$\times$};
  \node at (3.5,0.5) [text=red] {$\times$};
  \node at (3.5,1.5) [text=red] {$\times$};
  \node at (3.5,3) [text=red] {$\times$};
  \node at (2.5,0) [text=red] {$\times$};

  \node at (5,0) [highlight,label=left:{$\frac{1}{1}$}] {};
  \node at (5,1) [highlight,label=left:{$\frac{1}{1}$}] {};
  \node at (5,2) [highlight,label=left:{$\frac{0}{2}$}] {};
  \node at (5,3) [highlight,label=left:{$\frac{2}{0}$}] {};

  \node at (4.5,0) [text=red] {$\times$};
  \node at (4.5,0.5) [text=red] {$\times$};
  \node at (4.5,2.5) [text=red] {$\times$};
  \node at (4.5,2) [text=red] {$\times$};
  \node at (3.5,1) [text=red] {$\times$};

  \node at (6,0) [highlight,label=left:{$\frac{1}{1}$}] {$\frac44$};
  \node at (6,1) [highlight,label=left:{$\frac{1}{1}$}] {$\frac44$};
  \node at (6,2) [highlight,label=left:{$\frac{0}{2}$}] {$\frac33$};
  \node at (6,3) [highlight,label=left:{$\frac{2}{0}$}] {$\frac51$};

  \node at (5.5,0) [text=red] {$\times$};
  \node at (5.5,0.5) [text=red] {$\times$};
  \node at (5.5,2.5) [text=red] {$\times$};
  \node at (5.5,2) [text=red] {$\times$};
  \node at (4.5,1) [text=red] {$\times$};

  % Inputs and Outputs
  \node at (-0.5,4) [left, align=right] {Входная\\пара};

  \trellisIn{0}{11}
  \trellisIn{1}{00}
  \trellisIn{2}{00}
  \trellisIn{3}{01}
  \trellisIn{4}{10}
  \trellisIn{5}{10}
\end{tikzpicture}


Наложив полученный путь на решетку кодера, узнаем декодированное слово.
$\overline{m}_{получ}=101111001$

\end{document}
